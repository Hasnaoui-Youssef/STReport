\chapter*{Conclusion}
\addcontentsline{toc}{chapter}{Conclusion}
\markboth{Conclusion}{}

This project aimed to develop a demonstration for the "CryptoEngine", a recent addition to the STM32 security portfolio. Conducted under the guidance of the ST Support Solutions team, our primary goal was to create a comprehensive yet accessible guide for customers interested in adopting this advanced technology. We set out to offer a practical example that balances showcasing the new features of the "CryptoEngine" while remaining simple enough to serve as an introduction to cryptography on STM32 microcontrollers.

Our objectives centered around providing a framework for constructing a secure encrypted communication system using the "CryptoEngine" 
 while illustrating the enhanced capabilities of this technology compared to existing solutions and promoting a deeper understanding of cryptographic principles through practical application. By the end of the project, we achieved these objectives by successfully implementing various cryptographic techniques, including symmetric encryption, key exchange mechanisms, and digital signatures, directly on STM32 hardware.

From a personal perspective, this project has provided an invaluable introduction to the fields of cryptography and embedded security. The experience has strengthened my understanding of cryptographic techniques and allowed me to apply this knowledge in practical settings, gaining insight into how these techniques protect data in embedded systems. I also gained a clearer perspective on the intricacies of secure communications, which is crucial in today's increasingly interconnected and digital world.

Furthermore, this project has served as a unique opportunity to understand the workflow within a company specializing in microcontrollers, revealing the complexities and nuances of working in such an environment. It has offered a closer look at the challenges and rewards of developing secure embedded systems, and has underscored the importance of collaboration, precise documentation, and iterative testing in delivering robust, reliable solutions.

Overall, this project has been both a professional and personal milestone, serving as a solid foundation for further exploration and specialization in cryptography and embedded security.